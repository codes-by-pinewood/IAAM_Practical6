\documentclass{article}
% Language setting
% Replace `english' with e.g. `spanish' to change the document language
\usepackage[english]{babel}

\usepackage{caption}
% Set page size and margins
% Replace `letterpaper' with `a4paper' for UK/EU standard size
\usepackage[letterpaper,top=2cm,bottom=2cm,left=3cm,right=3cm,marginparwidth=1.75cm,margin=1in]{geometry}
\usepackage{listings}

% Useful packages
\usepackage{amsmath}
\usepackage{graphicx}
\usepackage{subfig}
\usepackage{algorithm}
\usepackage{algpseudocode}

\usepackage{graphicx}
\usepackage{caption}


\usepackage[colorlinks=true, allcolors=blue]{hyperref}

\title{3D Shape, Pre-processing and Visualization}
\author{Megan Mirnalini Sundaram R, Sherry Usman}

\begin{document}
\maketitle

\section*{Question 1}
\subsection*{Slices and Aspect Ratio}
The image has 16 slices. The aspect ratio can be defined as the ratio of total width (total number of columns) and the total height (total number of rows). In that case it is 160/140 which is 1.143. 

\subsection*{Function to display the contents of the 3D image} \label{sec:contents-3D_image}
The code snippet below shows the function for displaying the slices of the image.  
\begin{lstlisting}
import numpy as np
import matplotlib.pyplot as plt 

num_slices = 16
fig, axes = plt.subplots(4,4, figsize =(12,12))
for i in range(num_slices):
  ax = axes[i // 4, i % 4]
  tif.seek(i)
  slice = np.array(tif)

  ax.imshow(slice)
  ax.axis('off')
  ax.set_title(f'Slice {i+1}')

plt.tight_layout()
plt.show()
\end{lstlisting}

The results of this function are show in Appendix: Figure \ref{fig:3d-plane-image}. 
\section*{Question 2}
\subsection*{Thresholding and best value}
We applied thresholding on the chromo.tiff image using the diplib library in Python and using VAA3D software. Figure \ref{fig:Otsu} shows the result of the Otsu Threshold (left) and the result of Otsu threshold with median filter (right). 

\begin{figure}%
    \centering
    \subfloat[\centering Otsu Thresholding]{{\includegraphics[width=0.45\textwidth]{Report/Images/otsu_threshold_image.png} }} \label{fig:Otsu}%
    \qquad
    \subfloat[\centering Median Filter and Otsu Thresholding]{{\includegraphics[width=0.45\textwidth]{Report/Images/median_filter.png} }}%
    \caption{Chromo.tiff with Otsu threshold and Median filter and Otsu threshold}%
    \label{fig:Otsu}%
\end{figure}

Figure \ref{fig:vaa3dthresh} shows the result of the Vaa3D software. As seen in figure \ref{fig:vaa3dthresh} the best threshold value is 55. 

\begin{figure}[h!]
    \centering
    \includegraphics[width=1\linewidth]{Report/Images/vaa3d_thresh.png}
    \caption{Chromo.tiff image thresholded in VAA3D}
    \label{fig:vaa3dthresh}
\end{figure}



\subsection*{Algorithm for Depth Cuing}
The figure \ref{fig:depth-cuing} shows the algorithm for depth cuing. As shown in the algorithm 16 evenly spaced values are taken from the range 0 to 255 and the slices are thresholded with the values. Slices lower in the plane are thresholded with lower values and slices higher in the plane are thresholded iwth higher values. %with higher   
\begin{figure}
    \centering
    \includegraphics[width=0.5\linewidth]{Report/Images/depth_cuing.png}
    \caption{Algorithm for depth cuing }
    \label{fig:depth-cuing}
\end{figure}




\subsection*{Depth Cuing}
Figure \ref{fig:depthcuing} shows the result of depth cuing. 
\begin{figure}
    \centering
    \includegraphics[width=1\linewidth]{Report/Images/mip.png}
    \caption{Chromo.tiff with depth cuing}
    \label{fig:depthcuing}
\end{figure}

The first image is a raw image obtained before processing techniques such as thresholding, median filter and contrast stretching. Thus it looks out of focus, blurry and underexposed. The second image is the product of these image analysis methods and the original image. 




\section*{Question 3}
\subsection*{Maximum Projection and Alpha Function}
As instructed in the file, the contrast was varied. In addition to this, the thickness \textbf{Z-thick} was also varied, to understand the effect on the contrast. \\The video files were generated for both maximum projection and alpha function, and are submitted along with this file. 

\medskip

\textbf{Maximum Projection} : File Name - \textit{chromo-MIP\_Contrast.mp4}

\textbf{Alpha Projection} : File Name - \textit{chromo-Alpha\_Contrast.mp4}
\subsection*{Surface Visualization}

\begin{figure}
    \centering
    \includegraphics[width=0.3\linewidth]{Report/Images/6.3.2/0-75,25.png}
    \includegraphics[width=0.3\linewidth]{Report/Images/6.3.2/0-75,25-rotated.png}
    \includegraphics[width=0.3\linewidth]{Report/Images/6.3.2/0-75,25-sliced.png}
    \caption{Threshold: 0-75, mesh density:25, original (left), rotated (middle), cross-section(right)}
    \label{fig:enter-label}
\end{figure}

\begin{figure}
    \centering
    \includegraphics[width=0.3\linewidth]{Report/Images/6.3.2/0-75,50.png}
    \includegraphics[width=0.3\linewidth]{Report/Images/6.3.2/0-75,50-rotated.png}
    \includegraphics[width=0.3\linewidth]{Report/Images/6.3.2/0-75,50-sliced.png}
    \caption{Threshold: 0-75, mesh density:50, original (left), rotated (middle), cross-section(right)}
    \label{fig:enter-label}
\end{figure}


\begin{figure}
    \centering
    \includegraphics[width=0.3\linewidth]{Report/Images/6.3.2/0-75,75.png}
    \includegraphics[width=0.3\linewidth]{Report/Images/6.3.2/0-75,75-rotated.png}
    \includegraphics[width=0.3\linewidth]{Report/Images/6.3.2/0-75,75-sliced.png}
    \caption{Threshold: 0-75, mesh density:75, original (left), rotated (middle), cross-section(right)}
    \label{fig:enter-label}
\end{figure}

\begin{figure}
    \centering
    \includegraphics[width=0.3\linewidth]{Report/Images/6.3.2/0-75,100.png}
    \includegraphics[width=0.3\linewidth]{Report/Images/6.3.2/0-75,100-rotated.png}
    \includegraphics[width=0.3\linewidth]{Report/Images/6.3.2/0-75,100-sliced.png}
    \caption{Threshold: 0-75, mesh density:100, original (left), rotated (middle), cross-section(right)}
    \label{fig:enter-label}
\end{figure}


\begin{figure}
    \centering
    \includegraphics[width=0.3\linewidth]{Report/Images/6.3.2/75-150,25.png}
    \includegraphics[width=0.3\linewidth]{Report/Images/6.3.2/75-150,25-rotated.png}
    \includegraphics[width=0.3\linewidth]{Report/Images/6.3.2/75-150,25-sliced.png}
    \caption{Threshold: 75-150, mesh density:25, original (left), rotated (middle), cross-section(right)}
    \label{fig:enter-label}
\end{figure}


\begin{figure}
    \centering
    \includegraphics[width=0.3\linewidth]{Report/Images/6.3.2/75-150,50.png}
    \includegraphics[width=0.3\linewidth]{Report/Images/6.3.2/75-150,50-rotated.png}
    \includegraphics[width=0.3\linewidth]{Report/Images/6.3.2/75-150,50-sliced.png}
    \caption{Threshold: 75-150, mesh density:50, original (left), rotated (middle), cross-section(right)}
    \label{fig:enter-label}
\end{figure}


\begin{figure}
    \centering
    \includegraphics[width=0.3\linewidth]{Report/Images/6.3.2/75-150,75.png}
    \includegraphics[width=0.3\linewidth]{Report/Images/6.3.2/75-150,75-rotated.png}
    \includegraphics[width=0.3\linewidth]{Report/Images/6.3.2/75-150,75-sliced.png}
    \caption{Threshold: 75-150, mesh density:75, original (left), rotated (middle), cross-section(right)}
    \label{fig:enter-label}
\end{figure}


\begin{figure}
    \centering
    \includegraphics[width=0.3\linewidth]{Report/Images/6.3.2/75-150,100.png}
    \includegraphics[width=0.3\linewidth]{Report/Images/6.3.2/75-150,100-rotated.png}
    \includegraphics[width=0.3\linewidth]{Report/Images/6.3.2/75-150,100-sliced.png}
    \caption{Threshold: 75-150, mesh density:100, original (left), rotated (middle), cross-section(right)}
    \label{fig:enter-label}
\end{figure}


\begin{figure}
    \centering
    \includegraphics[width=0.3\linewidth]{Report/Images/6.3.2/150-255,25.png}
    \includegraphics[width=0.3\linewidth]{Report/Images/6.3.2/150-255,25-rotated.png}
    \includegraphics[width=0.3\linewidth]{Report/Images/6.3.2/150-255,25-sliced.png}
    \caption{Threshold: 150-255, mesh density:25, original (left), rotated (middle), cross-section(right)}
    \label{fig:enter-label}
\end{figure}

\begin{figure}
    \centering
    \includegraphics[width=0.3\linewidth]{Report/Images/6.3.2/150-255,50.png}
    \includegraphics[width=0.3\linewidth]{Report/Images/6.3.2/150-255,50-rotated.png}
    \includegraphics[width=0.3\linewidth]{Report/Images/6.3.2/150-255,50-sliced.png}
    \caption{Threshold: 150-255, mesh density:50, original (left), rotated (middle), cross-section(right)}
    \label{fig:enter-label}
\end{figure}

\begin{figure}
    \centering
    \includegraphics[width=0.3\linewidth]{Report/Images/6.3.2/150-255,75.png}
    \includegraphics[width=0.3\linewidth]{Report/Images/6.3.2/150-255,75-rotated.png}
    \includegraphics[width=0.3\linewidth]{Report/Images/6.3.2/150-255,75-sliced.png}
    \caption{Threshold: 150-255, mesh density:75, original (left), rotated (middle), cross-section(right)}
    \label{fig:enter-label}
\end{figure}




\begin{figure}
    \centering
    \includegraphics[width=0.3\linewidth]{Report/Images/6.3.2/150-255,100.png}
    \includegraphics[width=0.3\linewidth]{Report/Images/6.3.2/150-255,100-rotated.png}
    \includegraphics[width=0.3\linewidth]{Report/Images/6.3.2/150-255,100-sliced.png}
    \caption{Threshold: 75-150, mesh density:100, original (left), rotated (middle), cross-section(right)}
    \label{fig:enter-label}
\end{figure}








\subsection*{Volume and Surface Visualization}

\section*{Question 4}
\subsection*{Static 3D Visualization}
\subsection*{Improving Signal Strength}
\subsection*{Automated Deconvolution}
\subsection*{Manual Deconvolution}
\subsection*{Visualizations}
\clearpage
\section*{Appendix}
\subsection*{Function to display image content in planes}
\begin{figure}[h!]
    \centering
    \includegraphics[width=1\linewidth]{Report/Images/3d_plot.png}
    \caption{The results of the function to display slices of tif image}
    \label{fig:3d-plane-image}
\end{figure}
% \bibliographystyle{alpha}
% \bibliography{sample}

\end{document}